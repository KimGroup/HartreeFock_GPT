\documentclass[showpacs,preprintnumbers,amsmath,amssymb,prl,aps,onecolumn,superscriptaddress]{revtex4}

\usepackage{graphicx}% Include figure files \usepackage{dcolumn}% Align table columns on decimal point 
\usepackage{bm}% bold math
\usepackage[usenames,dvipsnames]{color} \usepackage{ulem}

\newcommand{\bk}{{\bf k}} 
\newcommand{\bq}{{\bf q}}
\newcommand{\br}{{\bf r}} 
\newcommand{\bp}{{\bf p}}
\newcommand{\ep}{{$e$-ph}\:}
\newcommand{\ee}{{$e$-$e$}\:}
\newcommand{\needref}{\textcolor{red}{[ref]}}
\newcommand{\ligand}{{\underbar L}}
\renewcommand{\thefigure}{S\arabic{figure}}
\renewcommand{\theequation}{S\arabic{equation}}
\begin{document}
%\preprint{APS/123-QED}
\bibliographystyle{prsty}

\title{Charge disproportionation without charge transfer in the rare-earth 
nickelates as a possible mechanism for the metal-insulator transition - 
supplementary material}

\author{Steve Johnston}
\affiliation{Department of Physics and Astronomy, University of British Columbia, Vancouver, British Columbia, Canada V6T~1Z1}
\affiliation{Quantum Matter Institute, University of British Columbia, Vancouver, British Columbia, Canada V6T~1Z4}

\author{Anamitra Mukherjee}
\affiliation{Department of Physics and Astronomy, University of British Columbia, Vancouver, British Columbia, Canada V6T~1Z1}

\author{Ilya Elfimov}
\affiliation{Department of Physics and Astronomy, University of British Columbia, Vancouver, British Columbia, Canada V6T~1Z1}
\affiliation{Quantum Matter Institute, University of British Columbia, Vancouver, British Columbia, Canada V6T~1Z4}

\author{Mona Berciu}
\affiliation{Department of Physics and Astronomy, University of British Columbia, Vancouver, British Columbia, Canada V6T~1Z1}
\affiliation{Quantum Matter Institute, University of British Columbia,
Vancouver, British Columbia, Canada V6T~1Z4}

\author{George A. Sawatzky}
\affiliation{Department of Physics and Astronomy, University of British Columbia, Vancouver, British Columbia, Canada V6T~1Z1}
\affiliation{Quantum Matter Institute, University of British Columbia,
Vancouver, British Columbia, Canada V6T~1Z4}
\affiliation{Department of Chemistry, University of British Columbia, Vancouver, British Columbia, Canada V6T~1Z1}

\date{\today}

\pacs{71.30.+h, 71.38.-k, 72.80.Ga} \maketitle 

% 71.30.+h - metal insulator transitions
% 71.38.-k - electron-phonon interactions, electronic structure of solids.
% 72.80.Ga - Transition-metal compounds, electrical conductivity of
\section{Phonon displacement pattern}
In order to verify that the rock-salt-like distortion pattern assumed
in the main text is energetically favored we performed a series of ED
calculations for the Ni$_2$O$_{10}$ cluster. In each case we allowed
the ten O atoms to be displaced by $\delta d = \pm 0.1$
$\AA$ along the Ni-O bond direction. ED calculations for the ground
state energy were then carried out for the $2^{10}$ possible
configurations. The lowest energies were obtained for the two
displacement patterns corresponding to a contraction/expansion or
expansion/contraction of the O octahedra about the two Ni sites as 
considered in the article.

\section{Hartree-Fock Calculations}
We performed unbiased Hartree-Fock calculations for the model
Hamiltonian given in the main text.   
Our treatment follows that of Ref. \onlinecite{MeanField}, which we 
summarize here for completeness. We work in momentum space where the 
microscopic Hamiltonian can be rewritten in the form 
\begin{equation}
H=\sum_{\bk,\alpha,\beta,\sigma} T_{\alpha,\beta}(\bk) d^\dagger_{\bk,\alpha,\sigma}
d^{\phantom\dagger}_{\bk,\beta,\sigma} 
+ \sum_{\alpha,\alpha^\prime,\beta,\beta^\prime,\sigma,\sigma^\prime}\sum_{\bk,\bk^\prime,\bq} 
U^{\sigma,\sigma^\prime}_{\alpha,\alpha^\prime,\beta,\beta^\prime}(\bq)  
d^{\dagger}_{\bk,\alpha,\sigma}d^{\dagger}_{\bk^\prime,\alpha^\prime,\sigma^\prime}
d^{\phantom\dagger}_{\bk^\prime-\bq,\beta^\prime,\sigma^\prime}
d^{\phantom\dagger}_{\bk+\bq,\beta,\sigma}. 
\end{equation}
where the interaction on a Ni site is given by  
\begin{eqnarray}
U^{\sigma\sigma^\prime}_{\alpha,\alpha^\prime,\beta,\beta^\prime}&=& 
 \frac{U}{2} \delta_{-\sigma,\sigma^\prime}\delta_{\alpha,\alpha^\prime}\delta_{\alpha\beta}\delta_{\alpha\beta^\prime}  
 +\frac{U^\prime}{2}(1-\delta_{\alpha\alpha^\prime}) \delta_{\alpha\beta}\delta_{\alpha^\prime\beta^\prime} \\\nonumber
 &+&\frac{J}{2}(1-\delta_{\alpha\alpha^\prime})\delta_{\alpha\beta^\prime}\delta_{\alpha^\prime\beta} + 
 \frac{J^\prime}{2}\delta_{\alpha\alpha^\prime}\delta_{\beta\beta^\prime}(1-\delta_{\sigma\sigma^\prime})(1-\delta_{\alpha\beta}).
\end{eqnarray} 
The interaction on an O site is like the first term with $U$ replaced by $U_p$. 

Our aim is to now construct a HF Hamiltonian of the form  
\begin{equation}
    H_{\rm HF} = \sum_{n,\bk} E_n(\bk)\gamma^\dagger_{n,\bk,\sigma}\gamma^{\phantom\dagger}_{n,\bk,\sigma}, 
\end{equation}
where $n$ is a band index and $\bk$ is the electron momentum
restricted to the first Brillouin zone. 
We begin by expanding the HF operators in terms of the original basis set 
$\gamma_{\bk,n,\sigma} = \sum_{\alpha}
c_{n,\alpha}(\bk)d_{\bk,\alpha,\sigma}$, and then obtain the HF matrix
equation for the expansion coefficients   
$\sum_{\alpha} M_{\alpha\beta}(\bk) c_{\alpha}(\bk) = E_n(\bk)c_\alpha(\bk)$, 
which is solved self-consistently. The matrix $M_{\alpha\beta}(\bk)$
results from the equation:
\begin{equation}
   \langle \big\{[\gamma^{\phantom\dagger}_{n,\bk,\sigma},H], d^\dagger_{\bk,\alpha^\prime,\sigma^\prime}\big\} \rangle= 
    \big\{ E_n(\bk)\gamma^{\phantom\dagger}_{n,\bk,\sigma},d^\dagger_{\bk,\alpha^\prime,\sigma^\prime} \big\}.
\end{equation} 
For a quadratic Hamiltonian the left-hand side result of the
commutators/anticommutators is a c-number and no average would be
needed. For a quartic Hamiltonian that side contains operators which
are replaced with their HF values. This method leads to  the true
unbiased HF solution
(i.e., the Slater determinant that minimizes the total energy, see
\cite{MeanField}). 

After lengthy but straightforward algebra we find:
\begin{eqnarray}
M_{\alpha\beta}(\bk)=T_{\alpha,\beta}(\bk)&+&\frac{1}{2}\sum_{\bq,\alpha^\prime,\beta^\prime,\sigma^\prime} 
    \left[ U^{\sigma\sigma^\prime}_{\beta,\alpha^\prime,\alpha,\beta^\prime} + 
           U^{\sigma^\prime\sigma}_{\alpha^\prime,\beta,\beta^\prime,\alpha}
    \right] \langle d^\dagger_{\bq,\alpha^\prime,\sigma^\prime}d^{\phantom\dagger}_{\bq,\beta^\prime,\sigma^\prime} \rangle 
    \\ \nonumber 
 &-&\frac{1}{2}\sum_{\bq,\alpha^\prime,\beta^\prime} 
    \left[U^{\sigma\sigma}_{\alpha^\prime,\beta,\alpha,\beta^\prime} + 
          U^{\sigma\sigma}_{\beta,\alpha^\prime,\beta^\prime,\alpha}\right]
    \langle d^\dagger_{\bq,\alpha^\prime,\sigma}d^{\phantom\dagger}_{\bq,\beta^\prime,\sigma} \rangle 
\end{eqnarray}
where the average $\langle  \rangle$ defines the
self-consistent HF fields, which  are found iterationally. 
 It should be noted that the dependence on the static lattice displacement is 
introduced through the kinetic term $T_{\alpha,\beta}(\bk)$, see main text.  


\section{Comparison with Exact Diagonalization}
Here we compare the results of the HFA with those obtained from 
exact diagonalization for the Ni$_2$O$_6$ cluster where periodic
boundary conditions  
have been assumed such that only the $\bk = 0$ momentum point is included. 
Fig. \ref{Fig:MF_vs_ED}a shows a comparison of the ground state energy
(electronic component only), identical 
to the results shown in Fig. 3(a) of the main text.  
In general the HF treatment overestimates the exact answer by $\sim 1$ eV, 
however the overall dependence on displacement is well reproduced.  
This general behavior also holds for the total hole occupancies, shown in 
Fig. \ref{Fig:MF_vs_ED}b and \ref{Fig:MF_vs_ED}c for the exact and 
HF solutions, respectively. This comparison gives us confidence that the 
HF method can describe the physics of the nickelates for our choice in 
parameters. 

\begin{figure}
 \includegraphics[width=0.5\columnwidth]{ED_vs_MF.pdf}
 \caption{\label{Fig:MF_vs_ED} (color online)
 (a) The ground state energy per Ni atom of the 
 Ni$_2$O$_6$ cluster with periodic boundary conditions. 
 HF (red/dashed) and ED (black/solid) results are shown.
 (b) The total hole occupancy of the compressed Ni (black/solid), 
 expanded Ni (red/dash-dot), and O (green/dashed) sites as obtained 
 from ED. 
 (c) As panel (b) but obtained within the HF treatment.  
 }
\end{figure}

\begin{thebibliography}{99}
\bibitem{MeanField}
J.-P. Blaizot and G. Ripka, {\it Quantum Theory of Finite Systems}. The MIT Press (1986).
\end{thebibliography}
\end{document}

