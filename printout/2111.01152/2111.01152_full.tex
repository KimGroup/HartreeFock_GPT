\documentclass{article}
\usepackage{amsmath}
\usepackage{breqn}
\usepackage{bm}
\usepackage{fontspec}
\setmainfont{DejaVu Serif}
\usepackage{physics}
\usepackage{geometry}
\geometry{top=0.5cm,bottom=0.5cm,left=2cm,right=2cm}
\setlength{\parindent}{0pt}

% \usepackage[texMathDollars,tightLists]{markdown}
\usepackage[texMathDollars,pipeTables=true,hybrid,]{markdown}


% \def\markdownRendererInlineMath#1{\begin{math}#1\end{math}}
\def\markdownRendererDisplayMath#1{\begin{dmath*}#1\end{dmath*}}


\begin{document}
\sloppy

\markdownInput{2111.01152_full.md}

\end{document}

%%%%%%%%%%%%%%%%%%%%%%%%%%%%%%%%%%%%%%%%%%%%%%%%%%%%%%%%%%%%%%%
%
% Welcome to Overleaf --- just edit your LaTeX on the left,
% and we'll compile it for you on the right. If you open the
% 'Share' menu, you can invite other users to edit at the same
% time. See www.overleaf.com/learn for more info. Enjoy!
%
%%%%%%%%%%%%%%%%%%%%%%%%%%%%%%%%%%%%%%%%%%%%%%%%%%%%%%%%%%%%%%%
% \documentclass{article}
% \title{Using the markdown package}
% \usepackage[render]{markdown}
% \begin{document}

% \markdownInput{example.md}

% \end{document}
% \documentclass{article}
% % \usepackage{amsmath}
% \usepackage{breqn}
% \usepackage{bm}
% \usepackage[texMathDollars,pipeTables=true,hybrid]{markdown}
% \usepackage{geometry}
% \geometry{top=0.5cm,bottom=0.5cm,left=2cm,right=2cm}

% \def\markdownRendererDisplayMath#1{\begin{dmath}#1\end{dmath}}
% \begin{document}
% \sloppy
% \markdownInput{intro.md}
% \markdownInput{tab1.md}

% \end{document}
